
%%%%%%%%%%%%%%%%%%%%%%%%%%%%%%%%%%%%%%%%%%%%%%%%%%
\chapter{Introduction}
\label{chap:intro}

\epigraph{%
    Is there enough energy to move the entire current human population off-planet?
}{\textit{Adam}}

There are a bunch of science fiction movies where,
because of pollution, overpopulation, or nuclear war, humanity abandons Earth.
But lifting people into space is hard.
Barring a massive reduction in the population,
is launching the whole human race into space physically possible?
(See \cref{fig:the-plan}.)
Let's not even worry about where we're headed---%
we'll assume we don't have to find a new home, but we can't stay here.

\begin{definition}{Baseline energy}{baseline-energy}
The \emph{baseline energy per person} is the minimum energy required to reach Earth escape velocity.
\end{definition}

\begin{theorem}{Energy lower bound}{energy-lower-bound}
Under idealized assumptions (no drag, no inefficiencies), the required energy per person is at least $\tfrac12 m v_\text{esc}^2$.
\end{theorem}

We will refer back to \cref{def:baseline-energy} and \Cref{thm:energy-lower-bound} throughout \cref{chap:analysis}.

\begin{figure}[hp]
\centering
\fbox{\rule{0pt}{0.28\textheight}\rule{0.9\textwidth}{0pt}}
\caption{Placeholder illustration for \cref{fig:the-plan}.}
\label{fig:the-plan}
\end{figure}
